\documentclass[12]{extarticle}

\usepackage{amsmath}
\usepackage{parskip}
\usepackage{graphicx}
\usepackage{amsfonts}
\usepackage{algpseudocode}

\begin{document}

\title{Introduction to Neuro-Evolution: Solutions}
\author{Andrew Geng, Jonas Klare, John Balis}




\maketitle





\section{Simulated Annealing}

\subsection{}

\subsection{}

\subsection{}

\section{Mutation and Speciation}


\section{Neuro-Evolution}
\subsection{A}
Possible observations: 

1.none of the training runs resulted in significant survival time
2. One or more training run resulted in signficant survival time, and can be seen to exhibit intelligent some sort of behavior with the intent of avoiding the hunter. 

It's important to note that neuro-evolution is very unpredictable, and may fail to find a good solution even after a very long time. However, running a given training run for longer can only improve the performance of the network.


\subsection{B}
 You should observe that the sample network has a very effective a deliberate style of avoiding the hunter. All successful SSNs we have trained have employed a variant of this style. It's important to note that while is no upper bound on how long it takes to train a network to this quality and we could not leave it's generation as an excercise, it is possible to train a network to this quality in even 1000 or fewer epochs. 


\subsection{C}

Possible Obervations:



\section{Works Cited}




\end{document}

